\section{Fazit}
\subsection{Soll/Ist-Vergleich}
Beim Soll/Ist-Vergleich ist vor allem aufgefallen, dass die Implementierung und Dokumentation deutlich länger gedauert hat als angenommen.\\
Die Anforderungen, die während der, kürzer als geplant ausgefallenen, Analyse und im Entwurf gestellt wurden, sind zu etwa 90 Prozent erfüllt worden.\\


\begin{center}
\begin{table} [htb] % table hat mehr Eigensachaften, z.B. Caption
\centering
\begin{tabular}{  l  r  }
\hline
\rowcolor{cyan}\textbf{Projektphase}&\textbf{geplante Zeit}\\
\hline
Analyse&4h\\
\rowcolor{lightgray}Entwurf&4h\\
Implementierung&24h\\
\rowcolor{lightgray}Dokumentation&8h\\
\hline
\textbf{Gesamt}&\textbf{40h}\\
\hline
\end{tabular}
\caption{Projektphasen}
\label{tab:Projektphasen}
\end{table}
\end{center}


\subsection{Lesson Learned}

Abschließend kann ich feststellen, dass ein Design mit dem Window- Builder und Eclipse nicht zu empfehlen ist. Dadurch, dass alle Anforderungen an ein Benutzfreundliches Design sehr Zeitaufwendig im Gegensatz zu anderen Entwicklungsumgebungen sind, dient es meiner Meinung nach nur für kleinere Oberflächen für Zweck Programme die wenige Eingaben erfordern.\\
Gut hervorheben kann ich die einfache Einbundung von HTTP und SSL Klassen und die verwendung von SQLite Datenbanken. Das hat sehr gut funktioniert.
