
\section{Analyse}

\subsection{Anforderung an das Programm}
Es wird ein neues Programm von Beginn an entwickelt.\\
Das Programm soll eine Bücherliste tabellarisch darstellen.
Jedes Buch soll mehrere Eigenschaften besitzen und diese sollen mit dargestellt werden.
\\
Dem Programm soll eine Datenbank zugrunde liegen aus der zu Programmstart geladen wird.\\
Über eine Maske sollen neue Bücher hinzugefügt werden können und ausgewählte Bücher müssen gelöscht werden können. Die einzelnen Eigenschaften der Felder müssen geändert werden können.
\\ 
Über ein Eingabefeld soll eine Suche nach der ISBN Nummer möglich sein.\\
Gefundene Ergebnisse sollen angezeigt werden und bei bedarf der Bücherliste hinzugefügt werden können.\\

\subsubsection{Eigenschaften eines Buches}

\begin{itemize}
\item{Nummer in der Liste}
\item{Name des Buches}
\item{Name des Autors}
\item{Stichwort/ Genre}
\item{Jahr der Erstauflage}
\item{ISBN Nummer}
\item{Ort/Platz im Regal}

\end{itemize}

\subsection{Anwendungsfälle}
Es soll für einen Einzelplatz für einen Anwender mit einer privaten Büchersammlung entwickelt werden.

\subsection{Projektbegründung}
Bei einer Büchersammlung ist es Sinnvoll einen geordneten Überblick zu haben.
Vor allem um Bücher schnell zu finden sowohl für Bücher im Bestand als auch für neue Bücher, die hinzugefügt werden sollen.

\subsection{Programmschnittstellen}
Die Datenbank als Speicher ist die erste Schnittstelle.\\
Eine geeignete online API zu einer Bücherdatenbank soll für die ISBN-Suche verwendet werden.

\subsection{Projektphasen}
\begin{center}
\begin{table} [htb] % table hat mehr Eigensachaften, z.B. Caption
\centering
\begin{tabular}{  l  r  }
\hline
\rowcolor{cyan}\textbf{Projektphase}&\textbf{geplante Zeit}\\
\hline
Analyse&8h\\
\rowcolor{lightgray}Entwurf&8h\\
Implementierung&16h\\
\rowcolor{lightgray}Dokumentation&8h\\
\hline
\textbf{Gesamt}&\textbf{40h}\\
\hline
\end{tabular}
\caption{Projektphasen}
\label{tab:Projektphasen}
\end{table}
\end{center}

\subsection{Entwicklungsprozess}
Bei der Bearbeitung des Projekts wird das Wasserfallmodel verfolgt.
Vom Problem zum Programm in folgenden Schritten:
\begin{itemize}
\item{Analyse}
\item{Entwurf}
\item{Implementierung}
\item{Test}
\end{itemize}